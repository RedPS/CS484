

\documentclass{article} 
%\usepackage[margin=1in]{geometry}
\setlength\topmargin{0pt}
\addtolength\topmargin{-\headheight}
\addtolength\topmargin{-\headsep}
\setlength\oddsidemargin{0pt}
\setlength\textwidth{\paperwidth}
\addtolength\textwidth{-2in}
\setlength\textheight{\paperheight}
\addtolength\textheight{-2in}

\usepackage[english]{babel}
\usepackage[utf8]{inputenc}
\usepackage{fancyhdr}
\usepackage{indentfirst}

\usepackage{changepage} 
\usepackage{booktabs}

% header
\pagestyle{fancy}
\fancyhf{}
\rhead{CS 484}
\lhead{Pedram Safaei}
\rfoot{Page \thepage}


\begin{document} 
\section*{\underline{VR Apps / Experiences}}

\subsection*{Google Earth VR}
Google Earth VR allows the users to travel to anywhere without needing to worry about the time conflicts or financial difficulties.  The virtual reality experience of Google Earth allows you to be anywhere you'd like; your hometown, the school you grew up in, the street you used to play in, and home you grew up in. With Google Earth VR it is you, on the places you'd like, and if you look off in another direction you will immediately see what is actually in that direction. \\
Experiencing Google Earth in VR is vastly different than experiencing in on a screen, for many reasons. One would be the fact that on a screen you'd see one perspective, a 2 dimensional picture, you would need a controller to turn, or in this case change the pictures or the angle. The virtual reality version of Google Earth uses your hands, your eyes, and your whole body as sort of a navigation system. In Google Earth VR you can turn and almost immediately see another angle of the place you've traveled to in the imaginary space. You can You can use your hands to zoom in, or in some cases you can use your whole body to walk towards something and get closer to it like how you would in the real world.\\
There is an option in Google Earth VR where the user can face the sky and change the time of day by just dragging and choosing. You can have sunset, darkness of night, or the beauty of midday. These are all just some of the factors to why Google Earth VR contributes greatly to what and how the user feels when using the VR experience of the app instead of the normal 2 dimensional version that is accessible on any screen. 



\subsection*{Beat Saber}
Beat Saber is a VR rhythm game where you would slash the beats for hours without feeling tired. Beat Saber relies on the movement of the player, but the movements have to be precise, it is like dancing mixed with swordplay. \\
Beat Saber gives the users the same feeling of other familiar games like Guitar Heroes or Rock Band but in a completely different experience. Old school games like Rock Band or Guitar Hero would require the user to have special equipment in the form of the instrument the user wanted to play or experience. The truth is that Beat Saber is no different when it comes to equipment but with Beat Saber users don't have to have equipment that imitate any instruments. \\
Beat Saber could even be categorized as an scary game. Users can see the beats coming towards their body and face and slashing them as quickly as they can is not just for gaining points, for some it could be a form of protection. When users don't slash the beats on time they can see the beats coming straight to their face or their body. Depending on the theme and the music the users will  experience vibration at the same time as the beats clash with their body. Small features like that will make you feel like you are in the environment and actually getting rid of real threats. 



\subsection*{Anatomyou}
This application uses virtual 3 dimensional navigation to show the anatomical structures of various systems of the human body. The application was geared toward health science students and people in the medical field, but it was later used to teach various students from different ranks knowledge about the human anatomy.\\
This application is actually unrealistic compared to other virtual reality applications as it will not give you the feeling of being a doctor or someone who is operating on human body. The application acts like a small camera that goes through different systems of the human body and will give you the feeling of being the camera. You can travel through different organisms and experience what they look like and what happens inside those systems. Again just like any other virtual reality application you can control the angle and travel through a 3 dimensional space and choose your route just by looking at a direction instead of using some sort of physical controller that would be required for a screen. \\
Some of the most amazing sensations will come from the fact that the user will be experiencing some of the conditions for the first time. Even movies can't really show what is happening inside of the body, and that is exactly what this application is geared towards. Seeing new sights with the first person point of view instead of looking at pictures or a movie will give users a feeling of discovery.

\newpage

\section*{\underline{Compare and Contrast}}




\end{document}

